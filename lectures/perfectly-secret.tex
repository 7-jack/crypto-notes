%! TeX root: ../notes.tex
\section{Perfectly-Secret Encryption}
\label{perfectly-secret}

Various notes from chapter 2 of \cite{katz-lindell}.

\begin{definition}[Perfectly Secret Encryption Scheme]
    Even an adversary with \textit{unbounded} computational power cannot break a \vocab{perfectly secret encryption scheme}.
\end{definition}

First, we must formally define an encryption scheme.
\begin{definition}[Encryption Scheme]
    Consists of three algorithms:
    \begin{itemize}
        \item $\Gen$, which outputs a key $k$ according to a distribution. The key space is denoted by $\KK$ and is finite.
        \item $\Enc(k, m)$, which encrypts $m$ under $k$. The space of possible ciphertexts is denoted by $\CC$.
        \item $\Dec(k, m)$, which decrypts $m$ under $k$.
    \end{itemize}
    And also a message space $\MM$ where $|\MM| > 1$.
\end{definition}

\begin{definition}[Perfectly Correct]
    For all $k \in \KK$, $m \in \MM$, if $c \sampledfrom \Enc_k(m)$ then $\Dec_k(c) = m$ with probability 1.        

    Unless stated otherwise, we will be working with perfectly correct encryption schemes.
\end{definition}

Note that we will reference distributions over $\KK, \MM$, and $\CC$. The distribution over the key space is given by $\Gen$. The distribution over the message space models how not all messages have equal probability of being sent.

$\KK$ and $\MM$ are independent distributions, as messages are chosen independent of keys. However, $\CC$ is fully determined by the distributions over $\KK$ and $\MM$. The distribution of $\KK$ is fixed for a given encryption scheme since it is defined by $\Gen$, but $\MM$ may vary depending on the parties. 

\begin{definition}[Perfect Secrecy]
    An encryption scheme $(\Gen, \Enc, \Dec)$ is \vocab{perfectly secret} if for all $m \in \MM, c \in \CC$ such that $\Pr[C = c] > 0$, we have 
    \[ \Pr[M = m \st C = c] = \Pr [M = m]\] % TODO: define a macro for s.t. |
\end{definition}

\begin{remark}
   Intuitively, this says that observing a ciphertext does not change the a priori probability of a message being sent. 
\end{remark}

\begin{remark}
    In the future we may omit conditions such as $\Pr[C = c] > 0$ simply for convention and ease of use; however, it is implicitly there. See exercise 2.6.
\end{remark}

\begin{lemma}
    An equivalent definition of perfect encryption states
    \[\Pr[C = c \st M = m] = \Pr[C = c] \]
\end{lemma}

Another equivalent and useful formulation of perfect secrecy is \textit{perfect indistinguishability}. % TODO: add stuff to the dictionary
\begin{definition}[Perfect Indistinguishability]
   \vocab{Perfect Indistinguishability} states that given $m_0, m_1 \in \MM$, then 
   \[\CC(m_0) = \CC(m_1) \]
   where $\CC(m_i)$ denotes the distribution of ciphertexts for the encryption of $m_i$.
\end{definition}

Phrased differently,
\begin{lemma}
An encryption scheme $\EncScheme$ over $\MM$ is perfectly secret iff for all $m_0, m_1 \in \MM, c \in \CC$,
\[\Pr[C = c \st M = m_0] = \Pr[C = c \st  M = m_1] \]
\end{lemma}

The final equivalent formulation of perfect secrecy is \textit{adversarial indistinguishability}.

\begin{definition}[Adversarial Indistinguishability]
    The intuition is that an adversary cannot distinguish between the encryption of two messages better than simply guessing. It is a taste of future "game based" definitions of security.

    We define an experiment $\text{PrivK}^\text{eav}$ in the setting of private key encryption and an eavesdropping adversary. It is defined for a scheme $\Pi = \EncScheme$ over $\MM$ for arbitrary $\AA$. One execution of the experiment is denoted $\text{PrivK}^\text{eav}_{\AA, \Pi}$ and defined as 
    \begin{enumerate}
        \item $\AA$ chooses a pair of messages $m_0, m_1 \in \MM$.
        \item $k \sampledfrom \KK$ and $b \sampledfrom {0, 1}$. $c := \Enc(m_b)$ and given to $\AA$.
        \item $\AA$ outputs  $b'$.
        \item If $b = b'$ then the output of the experiment is 1. Otherwise it is 0.
    \end{enumerate}
\end{definition}

\begin{lemma}[Adversarial Indistinguishability]
    An encryption scheme $\EncScheme$ over $\MM$ is perfectly secure if for all $\AA$,
    \[ \Pr[\text{PrivK}^\text{eav}_{\AA, \Pi}] = \frac{1}{2}. \]
\end{lemma}

\subsection{One Time Pad is Perfectly Secret}

The one time pad encryption scheme is given by the following algorithm:
\begin{enumerate}
    \item Fix an integer $l > 0$. Then $\MM, \KK, \CC$ are all equal to $\{0, 1\}^l$.
    \item $\Gen$ chooses a random string in $\KK$ uniformly.
    \item $\Enc_k(m) = c := m \xor k$.
    \item $\Dec_k(c) = m := c \xor k$
\end{enumerate}

Intuitively, the one time pad (proposed in 1917 by Verman) is perfectly secret because for any ciphertext, every plaintext could have been the original. Now since the keys are uniform randomly selected, each plaintext is equally likely. This intuition wasn't formally proved until 25 years later when Shannon introduced the idea of perfect secrecy.

\begin{theorem}
    The one-time pad encryption scheme is perfectly secure.
\end{theorem}
\begin{proof}
    We will aim to show that an adversary cannot do better than guessing which plaintext a ciphertext came from.

    Fix an arbitrary distribution over $\MM$ and choose $m \in \MM, c \in \CC$. Observe that
    \begin{align*}
        \Pr[C = c \st M = m] &= \Pr[M \xor K = c \st M = m]\\
                             &= \Pr[m \xor K = c]\\
                             &= \Pr[K = c \xor m]\\
                             &= \frac{1}{2^l}
    \end{align*}
    Where the last step holds because the key is chosen uniformly out of $\frac{1}{2^l}$ strings. Now since the distribution over messages was arbitrary, and the message itself was arbitrary, we see that 
    \[ \Pr[C = c | M = m_0] = \frac{1}{2^l} = \Pr[C = c | M = m_1]\]
    which proves perfect secrecy by Lemma 2.9.
\end{proof}

It is crucial to note that the length of the key must be equal to the length of the message, which in reality is impractical. Moreover, the key can only be used \textit{once}.


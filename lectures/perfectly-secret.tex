%! TeX root: ../notes.tex
\section{Perfectly-Secret Encryption}
\label{perfectly-secret}

Various notes from chapter 2 of \cite{katz-lindell}.

\begin{definition}[Perfectly Secret Encryption Scheme]
    Even an adversary with \textit{unbounded} computational power cannot break a \vocab{perfectly secret encryption scheme}.
\end{definition}

First, we must formally define an encryption scheme.
\begin{definition}[Encryption Scheme]
    Consists of three algorithms:
    \begin{itemize}
        \item $\Gen$, which outputs a key $k$ according to a distribution. The key space is denoted by $\mathcal{K}$ and is finite.
        \item $\Enc(k, m)$, which encrypts $m$ under $k$. The space of possible ciphertexts is denoted by $\mathcal{C}$.
        \item $\Dec(k, m)$, which decrypts $m$ under $k$.
    \end{itemize}
    And also a message space $\mathcal{M}$ where $|\mathcal{M}| > 1$.
\end{definition}

\begin{definition}[Perfectly Correct]
    For all $k \in \mathcal{K}$, $m \in \mathcal{M}$, if $c \sampledfrom \Enc_k(m)$ then $\Dec_k(c) = m$ with probability 1.        

    Unless stated otherwise, we will be working with perfectly correct encryption schemes.
\end{definition}

Note that we will reference distributions over $\mathcal{K}, \mathcal{M}$, and $\mathcal{C}$. The distribution over the key space is given by $\Gen$. The distribution over the message space models how not all messages have equal probability of being sent.

$\mathcal{K}$ and $\mathcal{M}$ are independent distributions. However, $\mathcal{C}$ is fully determined by the distributions over $\mathcal{K}$ and $\mathcal{M}$.


%! TeX root: ../notes.tex
\section{Introduction}
\label{introduction}

Various notes from \cite{katz-lindell}.

\begin{definition}[Kerchoffs' Principle]
    \vocab{Kerchoffs' Principle} states that the security of a cryptographic system  must not depend on the secrecy of the cipher.

    In other words, a cipher should be secure even if the algorithm is public.
\end{definition}

A few simple attack scenarios:
\begin{itemize}
    \item \vocab{Ciphertext-only attack}: The adversary observes ciphertexts and must try to determine the plaintext.
\item \vocab{Known-plaintext attack}: The adversary can learn one or more plaintexts encrypted under the same key. They must try to determine the plaintext of a different ciphertext they have not seen before.
\item \vocab{Chosen-plaintext attac}k: The adversary can learn encryptions of plaintexts of its choice. They must try to determine the plaintext of a different ciphertext like before.
\item \vocab{Chosen-ciphertext attac}k: The adversary can \textit{additionally} learn the plaintexts of ciphertexts of its choice. Their goal is the same as above.
\end{itemize}

Note that the first two are \vocab{passive} while the last two are \vocab{active}.

\begin{exercise}
    The first two scenarios are quite realistic. Can you think of some examples?
\end{exercise} 

\begin{exercise}
    Think of, or research, real world scenarios of the latter two attack scenarios.
\end{exercise}

Historical cryptographic ciphers are weak by modern standards, but they give us a few important lessons:
\begin{itemize}
    \item \vocab{Sufficient key space principle}: it is a necessary condition to have a large key space (the domain from which keys are chosen). Otherwise, we can brute force all the keys.
    \item \vocab{Designing secure ciphers is hard!}: there were many ciphers such as the Vigenere cipher that were insecure (for example, to cryptanalysis). It is the goal of modern cryptography to rigorously define and prove security.
\end{itemize}

Modern cryptography comes with a few principles:
\begin{itemize}
    \item \vocab{Clear, rigorous definitions}.
    \item \vocab{Clearly stated assumptions; the more minimal assumptions the better}.
    \item \vocab{Rigorous proofs of security with respect to principles 1 and 2}.
\end{itemize}

\begin{example}
    To take an example, we give an idea of just how hard it is to rigorously define a \textit{secure encryption scheme}.
    \begin{itemize}
        \item \vocab{Try 1: Secure if no adversary can find the key}

                But what if the adversary simply learns the plaintext?

        \item \vocab{Try 2: Secure if no adversary can learn the plaintext}

                But what if the adversary learns 50\% of the plaintext? Or the length of the plaintext? Is this definition clear enough? What "percentage" is okay to learn?

        \item \vocab{Try 3: Secure if the adversary cannot determine \textit{any} character of the plaintext}

                What if our plaintext is an integer like our salary and the adversary learns the \textit{range} of our salary? Surely this isn't what we wanted with an "encryption" scheme.

        \item \vocab{Try 4: Secure if adversary cannot derive any \textit{meaningful} information from the ciphertext}

                Close, but no cigar. What exactly does meaningful \textit{mean}? Our encryption scheme could be used in multiple different contexts and "meaningful" could have different meanings in each. For a \textit{definition}, this attempt is not enough.

        \item \vocab{Try 5: Secure if an adversary cannot compute \textit{any} function of the plaintext from the ciphertext}

                This is a rigorous definition: we have replaced "meaningful" with a more meaningful term.
            
    \end{itemize}
\end{example}
